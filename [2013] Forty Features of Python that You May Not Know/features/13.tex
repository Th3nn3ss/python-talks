% Author: Victor Terron (c) 2013
% License: CC BY-SA 4.0

\begin{frame}[fragile]{13. \large Comprensión de conjuntos y diccionarios}
  \begin{block}{}
  \centering
    En Python 2.7.x+ y 3.x+ podemos crear no sólo listas, sino también
    \structure{conjuntos} y \structure{diccionarios} por comprensión.
  \end{block}

  \scriptsize
  \begin{exampleblock}
    {Un conjunto con los múltiplos de siete <= 100:}
    \begin{lstlisting}
>>> {x for x in range(1, 101) if not x % 7}
set([98, 35, 70, 7, 42, 77, 14, 49, 84, 21, 56, 91, 28, 63])
    \end{lstlisting}
  \end{exampleblock}

  \begin{exampleblock}
    {Otra forma de hacerlo:}
    \begin{lstlisting}
>>> mult7 = set(x for x in range(1, 101) if not x % 7)
>>> len(mult7)
14
>>> 25 in mult7
False
    \end{lstlisting}
  \end{exampleblock}
\end{frame}

\begin{frame}[fragile]{13. \large Comprensión de conjuntos y diccionarios}
  \scriptsize
  \begin{exampleblock}
    {Un diccionario con el cubo de los números [0, 9]:}
    \begin{lstlisting}
>>> cubos = {x : x ** 3 for x in range(10)}
>>> cubos[2]
8
>>> cubos[8]
512
    \end{lstlisting}
  \end{exampleblock}

  \begin{exampleblock}
    {Otra forma de hacerlo:}
    \begin{lstlisting}
>>> dict((x, x ** 3) for x in range(5))
{0: 0, 1: 1, 2: 8, 3: 27, 4: 64}
    \end{lstlisting}
  \end{exampleblock}

  \small
  \begin{block}
    {\centering PEP 274: Dict Comprehensions [2001]}
    \centering \url{http://www.python.org/dev/peps/pep-0274/}
  \end{block}
\end{frame}
