% Author: Victor Terron (c) 2013
% License: CC BY-SA 4.0

\begin{frame}{7. Bloques de código}
  \small
  \begin{center}
    En Python los bloques de código se definen mediante el
    \structure{sangrado}, no utilizando palabras clave, como en
    Fortran, o llaves, como en C. Ya que hay a quien le desagrada, e
    incluso odia esta característica, gracias al módulo
    \structure{\_\_future\_\_} es posible habilitar el uso de las
    llaves para la delimitación de bloques.
  \end{center}

  \Large
  \begin{alertblock}{}
    \centering
    from \_\_future\_\_ import braces
  \end{alertblock}

  \vspace{0.5cm}

  \small
  \begin{block}{\centering Python White Space Discussion}
    \centering
    \url{http://c2.com/cgi/wiki?PythonWhiteSpaceDiscussion}
  \end{block}
\end{frame}

\begin{frame}[fragile]{7. Bloques de código}
  \begin{exampleblock}{}
    \small
    \begin{lstlisting}[escapechar=!]
>>> from __future__ import braces
  File "<stdin>", line 1
!\color{red}{SyntaxError: not a chance}
    \end{lstlisting}
  \end{exampleblock}

  \begin{figure}
    \centering
    \includegraphics[height=4.5cm]{pics/trollface.png}
  \end{figure}
\end{frame}

\begin{frame}[fragile]{7. Bloques de código}
  \begin{alertblock}{}
    \centering
    Pero, si \structure{de verdad} quieres usarlos, ¡hay una forma!
  \end{alertblock}

\begin{exampleblock}{}
    \small
    \begin{lstlisting}[escapechar=!]
if foo: !\color{blue}{\#}!{
    print "It's True"
!\color{blue}{\#}!}
else: !\color{blue}{\#}!{
    print "It's False"
!\color{blue}{\#}!}
    \end{lstlisting}
  \end{exampleblock}

  \small
  \begin{block}{\centering Is it true that I can't use curly braces in Python?}
    \centering
    \url{http://stackoverflow.com/a/1936210/184363}
  \end{block}
\end{frame}
