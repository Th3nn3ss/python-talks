% Author: Victor Terron (c) 2013
% License: CC BY-SA 4.0

\documentclass[14pt]{beamer}

\usepackage[utf8]{inputenc}
\usepackage{listings}
\usepackage[T1]{fontenc}

% A Nice Title Page for Beamer Presentations
% http://github.com/dfalster/BeamerTitleSlide
\usepackage{tikz}
\usepackage[framemethod=tikz]{mdframed}

% Define a new mdframed environment
\newmdenv[tikzsetting={draw=black,fill=white,fill opacity=0.65},
  backgroundcolor=none,leftmargin=0,rightmargin=0, innertopmargin=4pt,
  skipbelow=\baselineskip, skipabove=\baselineskip]{TitleBox}

% Reduce space between image and caption; also remove prefix 'Figure:'
% http://tex.stackexchange.com/a/94018
% http://tex.stackexchange.com/a/82462
\usepackage[font=small,skip=1pt,labelformat=empty]{caption}

% New line between paragraphs, no indentation
% http://tex.stackexchange.com/q/42
\usepackage[parfill]{parskip}

\usetheme{Copenhagen}
\useoutertheme{infolines}
\setbeamercovered{dynamic}
\setbeamertemplate{navigation symbols}{} % remove navigation symbols

\lstset{basicstyle=\ttfamily,language=python}

% Remove the space symbol for code inside the double quotation marks
% http://www.latex-community.org/viewtopic.php?f=4&t=248
\lstset{showstringspaces=false}

% Enable accents and ñ in our code listings
% http://stackoverflow.com/a/2782147/184363
\lstset{
  literate={á}{{\'a}}1
           {é}{{\'e}}1
           {í}{{\'i}}1
           {ó}{{\'o}}1
           {ú}{{\'u}}1
           {ñ}{{\~n}}1
           {¡}{{\textexclamdown}}1
}

% \input a file in the ./puntos/ directory
\newcommand{\feature}[1]{\input{./features/#1}}

\definecolor{red}{HTML}{F41B1B}
\definecolor{darkerred}{HTML}{DE2931}
\definecolor{blue}{HTML}{2E3CEC}
\definecolor{green}{HTML}{3C8031}

% Beamer: footer with current slide number but not total slide number
% http://tex.stackexchange.com/a/32815
\makeatletter
\setbeamertemplate{footline}
{
  \leavevmode%
  \hbox{%
  \begin{beamercolorbox}[wd=.333333\paperwidth,ht=2.25ex,dp=1ex,center]{author in head/foot}%
    \usebeamerfont{author in head/foot}\insertshortauthor~~\beamer@ifempty{\insertshortinstitute}{}{(\insertshortinstitute)}
  \end{beamercolorbox}%
  \begin{beamercolorbox}[wd=.333333\paperwidth,ht=2.25ex,dp=1ex,center]{title in head/foot}%
    \usebeamerfont{title in head/foot}\insertshorttitle
  \end{beamercolorbox}%
  \begin{beamercolorbox}[wd=.333333\paperwidth,ht=2.25ex,dp=1ex,right]{date in head/foot}%
    \usebeamerfont{date in head/foot}\insertshortdate{}\hspace*{2em}
    \insertframenumber\hspace*{2ex}
  \end{beamercolorbox}}%
  \vskip0pt%
}
\makeatother

\title{["Python"] * 40}
\author{Víctor Terrón}
\date{23 de noviembre de 2013}
\institute{IAA-CSIC}

\begin{document}

\section{Introducción}

{
\usebackgroundtemplate{\includegraphics[height=1.0\paperheight]{pics/un-palo.jpg}}

\begin{frame}[plain]
  \vspace{6.25cm}
  \begin{TitleBox}
    {\LARGE \inserttitle} \\
    {\small Cuarenta características de Python que \textcolor{darkerred}{quizás} no conoces \\
     \insertauthor \enspace --- \thinspace \url{http://vterron.xyz/forty}}
  \end{TitleBox}
\end{frame}
}

\input{intro.tex}

\begin{frame}{}
  \begin{alertblock}{}
    \centering \Large ¿Listos?
  \end{alertblock}

  \begin{figure}
    \centering
    \includegraphics[height=6cm]{pics/a-clockwork-orange.jpg}
  \end{figure}
\end{frame}

\section{Los diez primeros}

\feature{01.tex}
\feature{02.tex}
\feature{03.tex}
\feature{04.tex}
\feature{05.tex}
\feature{06.tex}
\feature{07.tex}
\feature{08.tex}
\feature{09.tex}
\feature{10.tex}

\section{Los diez siguientes}

\feature{11.tex}
\feature{12.tex}
\feature{13.tex}
\feature{14.tex}
\feature{15.tex}
\feature{16.tex}
\feature{17.tex}
\feature{18.tex}
\feature{19.tex}
\feature{20.tex}

\section{}

\begin{frame}{}
  \begin{block}{}
    \centering \Large Respira profundamente
  \end{block}

  \begin{figure}
    \centering
    \includegraphics[height=6cm]{pics/a-clockwork-orange-2.jpg}
  \end{figure}
\end{frame}

\section{Diez más}

\feature{21.tex}
\feature{22.tex}
\feature{23.tex}
\feature{24.tex}
\feature{25.tex}
\feature{26.tex}
\feature{27.tex}
\feature{28.tex}
\feature{29.tex}
\feature{30.tex}

\section{Recordatorio}
\begin{frame}[fragile]{A veces nos sentimos así}
  \begin{figure}
    \centering
    \includegraphics[height=0.75\paperheight]{pics/first-day-internet.jpg}
  \end{figure}
\end{frame}

\section{Los últimos diez}

\feature{31.tex}
\feature{32.tex}
\feature{33.tex}
\feature{34.tex}
\feature{35.tex}
\feature{36.tex}
\feature{37.tex}
\feature{38.tex}
\feature{39.tex}
\feature{40.tex}

\section{You Are Now a Ninja}
\begin{frame}{Programadores Python Shaolín}
  \begin{figure}
    \centering
    \includegraphics[height=7cm]{pics/shaolin.jpg}
  \end{figure}
\end{frame}

\end{document}
