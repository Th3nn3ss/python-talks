% Author: Victor Terron (c) 2013
% License: CC BY-SA 4.0

\begin{frame}[fragile]{11. Cadenas multilínea}

  \begin{block}{}
    \centering \large
    Una opción es usar \structure{barras invertidas}
  \end{block}

  \begin{exampleblock}{}
    \scriptsize
    \begin{lstlisting}
>>> parrafo = "En un lugar de la Mancha, de cuyo nombre " \
...           "no quiero acordarme, no ha mucho tiempo " \
...           "que vivía un hidalgo de los de lanza en " \
...           "astillero, adarga antigua, rocín flaco y " \
...           "galgo corredor."
>>> parrafo[:24]
'En un lugar de la Mancha'
>>> parrafo[-17:]
'y galgo corredor.'
    \end{lstlisting}
  \end{exampleblock}

  \begin{alertblock}{}
    \centering
    \small
    Pero, en el caso de trabajar con \structure{cadenas muy largas},
    puede ser molesto tener que \structure{terminar cada línea con
      \textbackslash}, sobre todo porque nos impide reformatear el
    párrafo con nuestro entorno de desarrollo o editor de texto ---
    ¡Emacs, por supuesto!
  \end{alertblock}
\end{frame}

\begin{frame}[fragile]{11. Cadenas multilínea}

  \begin{block}{}
    \centering
    Las comillas triples respetan \structure{todo} lo que hay dentro
    de ellas, incluyendo \structure{saltos de línea}. No es necesario
    escapar nada: todo se incluye directamente en la cadena.
  \end{block}

  \begin{exampleblock}{}
    \scriptsize
    %Escape triple quotes; otherwise string is italicized
    \begin{lstlisting}[escapechar=!]
>>> parrafo = !"""!En un lugar de la Mancha,
... de cuyo nombre no quiero
... acordarme
...
... !"""!
>>> print parrafo
En un lugar de la Mancha,
de cuyo nombre no quiero
acordarme


>>>
    \end{lstlisting}
  \end{exampleblock}
\end{frame}

\begin{frame}[fragile]{11. Cadenas multilínea}

  \begin{alertblock}{}
    \small
    \centering
    Una opción menos conocida es que podemos dividir nuestra cadena en
    líneas, una por línea, y \structure{rodearlas con paréntesis}.
    Python evalúa lo que hay dentro de los mismos y de forma
    automática las concatena:
   \end{alertblock}

  \begin{exampleblock}{}
    \scriptsize
    \begin{lstlisting}
>>> parrafo =  ("Es, pues, de saber, que este "
...             "sobredicho hidalgo, los ratos "
...             "que estaba ocioso")
>>> len(parrafo.split())
13
    \end{lstlisting}
  \end{exampleblock}

  \small
  \begin{exampleblock}
    {Un ejemplo más sencillo:}
    \begin{lstlisting}
>>> "uno" "dos"
'unodos'
    \end{lstlisting}
  \end{exampleblock}
\end{frame}
