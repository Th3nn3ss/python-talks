% Author: Victor Terron (c) 2013
% License: CC BY-SA 4.0

\begin{frame}[fragile]{15. pow(x, y, z)}
  \small
  \begin{block}{}
    \centering
    La función \structure{pow()} acepta, opcionalmente, un tercer
    argumento, \structure{z}. En caso de especificarse, la función
    devuelve \structure{pow(x, y) \% z}.  No es sólo práctico en
    algunos escenarios, sino que su implementación es más rápida que
    calcularlo nosotros en dos pasos.
  \end{block}

  \footnotesize
  \begin{exampleblock}{}
    \begin{lstlisting}
>>> pow(2, 3)
8
>>> pow(2, 3, 6)
2
>>> pow(2, 5, 17)
15
    \end{lstlisting}
  \end{exampleblock}

  \small
  \begin{block}
    {\centering Built-in Functions: pow()}
    \centering \url{http://docs.python.org/2/library/functions.html\#pow}
  \end{block}
\end{frame}
