% Author: Victor Terron (c) 2013
% License: CC BY-SA 4.0

\begin{frame}[fragile]{35. functools.partial()}
  \small
  \begin{block}{}
    \centering
    En el módulo \structure{functools}, la función
    \structure{partial()} nos permite \emph{congelar} otra función,
    \structure{fijándole argumentos} de forma que podamos llamarla de
    forma más sencilla y corta.
  \end{block}

  \begin{exampleblock}{}
    \begin{lstlisting}
>>> import functools
>>> eleva_dos = functools.partial(pow, 2)
    \end{lstlisting}
  \end{exampleblock}

\begin{exampleblock}
  {Y ahora usándola:}
    \begin{lstlisting}
>>> eleva_dos(3)
8
>>> eleva_dos(10)
1024
    \end{lstlisting}
  \end{exampleblock}
\end{frame}

\begin{frame}[fragile]{35. functools.partial()}
  \small
  \begin{alertblock}{}
    \centering
    Es importante tener presente que los argumentos que nosotros le
    pasamos a la función \emph{congelada} se añaden
    \structure{después} de los que hemos definido en
    \structure{partial()}. Es necesario dar un rodeo si queremos crear
    funciones parciales que añadan los argumentos por el comienzo --
    por ejemplo, para devolver \structure{pow(x, 3)}.
  \end{alertblock}

  \begin{block}
    {\centering Python: Why is functools.partial necessary?}
    \centering \url{http://stackoverflow.com/a/3252425/184363}
  \end{block}

  \begin{block}
    {\centering implementing functools.partial that prepends additional arguments}
    \centering \url{http://stackoverflow.com/q/11831070/184363}
  \end{block}
\end{frame}
