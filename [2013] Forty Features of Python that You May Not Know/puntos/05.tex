% Author: Victor Terron (c) 2013
% License: CC BY-SA 4.0

\begin{frame}[fragile]{5. import this}
  \Large
  \begin{block}{}
    \centering El Zen de Python
  \end{block}

  \begin{exampleblock}{\small Los principios filosóficos en los que se cimenta Python}
    \tiny
    \begin{lstlisting}
>>> import this
The Zen of Python, by Tim Peters

Beautiful is better than ugly.
Explicit is better than implicit.
Simple is better than complex.
Complex is better than complicated.
Flat is better than nested.
Sparse is better than dense.
Readability counts.
[...]
    \end{lstlisting}
  \end{exampleblock}

  \small
  \begin{block}{\centering PEP 20: The Zen of Python [2004]}
    \centering \url{http://www.python.org/dev/peps/pep-0020/}
  \end{block}
\end{frame}

\begin{frame}[fragile]{5. import this}
  \small
  \begin{center}
    Algo bastante más desconocido: el contenido del módulo
    \structure{this.py} está cifrado con el algoritmo
    \structure{ROT13} --- cada letra sustituida por la que está trece
    posiciones por delante en el alfabeto:
  \end{center}

  \begin{exampleblock}{}
    \tiny
    \begin{lstlisting}
>>> import inspect
>>> print inspect.getsource(this)
s = """Gur Mra bs Clguba, ol Gvz Crgref

Ornhgvshy vf orggre guna htyl.
Rkcyvpvg vf orggre guna vzcyvpvg.
Fvzcyr vf orggre guna pbzcyrk.
Pbzcyrk vf orggre guna pbzcyvpngrq.
Syng vf orggre guna arfgrq.
Fcnefr vf orggre guna qrafr.
Ernqnovyvgl pbhagf.
[...]
    \end{lstlisting}
  \end{exampleblock}

  \scriptsize
  \begin{block}{\centering this and The Zen of Python}
    \centering
    \url{http://www.wefearchange.org/2010/06/import-this-and-zen-of-python.html}
  \end{block}
\end{frame}
