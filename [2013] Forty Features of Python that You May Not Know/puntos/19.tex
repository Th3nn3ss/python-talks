% Author: Victor Terron (c) 2013
% License: CC BY-SA 4.0

\begin{frame}[fragile]{19. El operador ternario}
  \begin{center}
    El operator ternario (o condicional) nos permite escoger entre dos
    expresiones en \structure{función de una tercera}, la condición.
  \end{center}

  \begin{block}
    {\small
      En casi todos los lenguajes de programación tiene la misma forma:}
    \centering \LARGE test ? a : b
  \end{block}

  \begin{exampleblock}{Por ejemplo, en C:}
    \begin{lstlisting}[language=C]
int nsecs = (counter > 5) ? 5 : 0;
    \end{lstlisting}
  \end{exampleblock}
\end{frame}

\begin{frame}[fragile]{19. El operador ternario}
  \begin{center}
    En Python se conocen como \structure{expresiones condicionales}, y
    existen desde la versión 2.5. Su sintaxis es:
  \end{center}

  \begin{block}{}
    \centering \LARGE a if test else b
  \end{block}

  \begin{exampleblock}{}
    \begin{lstlisting}
>>> x = 2
>>> "uno" if x == 1 else "otra cosa"
'otra cosa'
    \end{lstlisting}
  \end{exampleblock}
\end{frame}

\begin{frame}[fragile]{19. El operador ternario}
  \footnotesize
  \begin{exampleblock}{}
    \begin{lstlisting}
>>> hora = 17
>>> msg = ("Es la" if hora == 1 else "Son las")
>>> msg + " {0}h".format(hora)
'Son las 17h'
    \end{lstlisting}
  \end{exampleblock}

  \begin{exampleblock}
    {Eleva al cuadrado si x es impar, divide entre dos de lo contrario:}
    \begin{lstlisting}
>>> x = 13
>>> y = x ** 2 if x % 2 else x / 2
>>> y
169
    \end{lstlisting}
  \end{exampleblock}

  \small
  \begin{block}
    {\centering PEP 308: Conditional Expressions [2003]}
    \centering \url{http://www.python.org/dev/peps/pep-0308/}
  \end{block}
\end{frame}
